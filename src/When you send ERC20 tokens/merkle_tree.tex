\begin{frame}[fragile]{Merkle Tree}
  \begin{tikzpicture}[scale=0.7,every node/.style={transform shape}]

\node[draw=none] at (0em, 0ex) {
\begin{tikzpicture}
\tikzset{every tree node/.style={align=center}}
\Tree [
  .$root=hash(hash(hash(A\|B)\|hash(C\|D))\|hash(hash(E\|F)\|hash(G\|H)))$ [
    .\textcolor{blue!50}{$hash(hash(A\|B)\|hash(C\|D))$} [
      .$hash(A\|B)$ $A$ $B$
    ][
      .$hash(C\|D)$ $C$ $D$
    ]
  ][
    .$hash(hash(E\|F)\|hash(G\|H))$ [
      .$hash(E\|F)$ \textcolor{blue!50}{$E$} \textcolor{red!50}{$F$}
    ][
      .\textcolor{blue!50}{$hash(G\|H)$} $G$ $H$
    ]
  ]
]
\end{tikzpicture}};

\node[align=left,anchor=north west] at (-20em,-15ex) {\textbf{Integrity}\\
root is changed if any leaf is changed.\\
You can compare roots only.};

\node[align=left,anchor=north west] at (-20em,-25ex) {\textbf{Inclusion Proof} (Merkle Proof)\\
If Alice knows a root already, you can convince her of the existence of any leaf, \textcolor{red!50}{$F$}.\\
You send her a Merkle proof: \textcolor{red!50}{$F$}, \textcolor{blue!50}{$E$} (left), \textcolor{blue!50}{$hash(G\|H)$} (right), \textcolor{blue!50}{$hash(hash(A\|B)\|hash(C\|D))$} (left).\\
Alice may compute a root with a Merkle proof only, and compare two roots.\\
You cannot fool her with nonexistent leaves.};

  \end{tikzpicture}
\end{frame}